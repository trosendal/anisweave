\documentclass{article}
\usepackage[letterpaper]{geometry}
\usepackage{animate}
\usepackage{graphicx}

\begin{document}

This is an example of using the animate latex package to generate
an animated figure in a PDF. The present method employs a strategy
'that builds the frames of the animation first in an r script and
then runs sweave/Latex to make the animation. The alternate strategy
isto generate the frames in the Sweave file. This is a strateg that
is described in R's animation package. It is suitable if your frames
are fast to build. If you have frames that take a long time to
generate then the present strategy allows you to build them first
and then run sweave later. This order is controlled in a makefile
that first builds the frames then the tex file then the pdf if the
frames are there it will
no rebuild them.

In this code:
\begin{verbatim}
\animategraphics[controls, loop,label=plot1]{10}{plot-}{1}{100}

'controls' tells latex that you want buttons below the animation
'loop' that you want it to repeat
'label' is you latex reference to this figure
{10} is the framerate
{plot-} is the common prefic of the files in the animation
{1}{100} are the running numbers of the file names, they are in numerical
order, not alphabetical as is the case when you make a .gif with imagemagik
\end{verbatim}


\animategraphics[controls, loop,label=plot]{10}{plot-}{1}{100}

\end{document}

%%This builds an animated figure from files named
%%plot-001,plot-002....plot-005 with controls below
%%and a frames rate of 3
